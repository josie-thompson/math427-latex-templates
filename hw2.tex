\documentclass{article}
\usepackage[utf8]{inputenc}
\usepackage[margin=1in]{geometry}

\usepackage{enumerate}

\usepackage{amsmath, amsfonts}
\usepackage{physics}

% fields
\newcommand{\C}{\mathbb{C}}
\newcommand{\R}{\mathbb{R}}
\newcommand{\Z}{\mathbb{Z}}

% helpful shorthand
\newcommand{\zbar}{\overline{z}}
\newcommand{\eps}{\varepsilon}
\newcommand{\suminf}[1]{\sum_{#1=1}^{\infty}}

% math functions
\newcommand{\argI}{\arg_I}
\newcommand{\logI}{\log_I}

\usepackage{graphicx}
\graphicspath{{./images/}}
\usepackage[export]{adjustbox}
\usepackage{float}

\newcommand{\Chapter}[1]{\section*{Chapter #1}}

\begin{document}
\Chapter{1.4}
\begin{enumerate}[1.]
% 1.4 #10
\setcounter{enumi}{9}
\item \begin{enumerate}[a.]
      % 1.4 #10.a
      \item {
            \newcommand{\z}{-i}
            \renewcommand{\a}{-\pi}
            \renewcommand{\b}{\pi}
            \newcommand{\I}{(\a, \b]}

            % example use of commands in this namespace:
            % \arg_{\I} \z
            }
      % 1.4 #10.b
      \item {
            \newcommand{\z}{-i}
            \renewcommand{\a}{0}
            \renewcommand{\b}{2\pi}
            \newcommand{\I}{[\a, \b)}
            }
      % 1.4 #10.c
      \item {
            \newcommand{\z}{1}
            \renewcommand{\a}{\frac{3\pi}{2}}
            \renewcommand{\b}{\frac{7\pi}{2}}
            \newcommand{\I}{[\a, \b)}
            }
      \end{enumerate}

% 1.4 #11
\item

% 1.4 #12 (optional)
%\item
\end{enumerate}

\Chapter{2.1}
\begin{enumerate}[1.]
% 2.1 #4
\setcounter{enumi}{3}
\item

% 2.1 #5
\item \begin{enumerate}[a.]
      % 2.1 #5.a
      \item
      % 2.1 #5.b
      \item
      % 2.1 #5.c
      \item
      \end{enumerate}

% 2.1 #7 (optional)
%\setcounter{enumi}{6}
%\item
\end{enumerate}

\Chapter{2.2}
\begin{enumerate}[1.]
% 2.2 #7
\setcounter{enumi}{6}
\item

% 2.2 #10
\setcounter{enumi}{9}
\item
\end{enumerate}
Show that $f(x+iy) = x^2$ is not an analytic function.
\end{document}
